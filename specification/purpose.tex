% -----------------------------------------------------------------------------
\section{Purpose}
% -----------------------------------------------------------------------------

People who engineer biological organisms often find it useful to draw diagrams in order to
communicate both the structure of the nucleic acid sequences that they are engineering
and the functional relationships between sequence features and other molecular species.
%
Some typical practices and conventions have begun to emerge for such
diagrams.  SBOL Visual aims to organize and systematize such
conventions in order to produce a coherent language for expressing
the structure and function 
of genetic designs. 
%
At the same time, we aim to make this language simple and easy to use,
allowing a high degree of flexibility and freedom in how such diagrams are organized, presented, and
styled---in particular, it should be readily possible to create
diagrams either by hand or using a wide variety of software programs.
%
Finally, means are provided for extending the language with new and
custom diagram elements, and for adoption of useful new elements into
the language.

\subsection{Relation to Data Models}

In order to ground SBOL Visual with precise definitions, we reference its visual elements to data models with well-defined semantics.
In particular, glyphs in SBOL Visual are defined in terms of their relation to the SBOL 2 data model (as defined in BBF RFC 112) and terms in the Sequence Ontology~\citep{SequenceOntology} and
the Systems Biology Ontology~\citep{SBO}.

SBOL Visual is not intended to represent designs at the same level of detail as these data models.
Effective visual diagrams are necessarily more abstract, focusing only on those aspects of a system that are the subject of the communication.
Nevertheless, we take as a principle that it should be possible to transform any SBOL Visual diagram into an equivalent (if highly abstract) SBOL 2 data representation.
Likewise, we require that SBOL Visual should be able to represent all of the significant structural or functional relationships in any GenBank or SBOL data representation.
