% -----------------------------------------------------------------------------
\section{SBOL Specification Vocabulary}
% -----------------------------------------------------------------------------

\subsection{Term Conventions}

% Note: yes, it's really RFC 0
This document indicates requirement levels using the controlled vocabulary specified in IETF RFC 2119 and reiterated in BBF RFC 0.
In particular, the key words "MUST", "MUST NOT", "REQUIRED", "SHALL", "SHALL NOT", "SHOULD", "SHOULD NOT", "RECOMMENDED", "MAY", and "OPTIONAL" in this document are to be interpreted as described in RFC 2119:

\begin{itemize}
\item The words "MUST", "REQUIRED", or "SHALL" mean that the item is an absolute requirement of the specification.
\item The phrases "MUST NOT" or "SHALL NOT" mean that the item is an absolute prohibition of the specification.
\item The word "SHOULD" or the adjective "RECOMMENDED" mean that there might exist valid reasons in particular circumstances to ignore a particular item, but the full implications need to be understood and carefully weighed before choosing a different course.
\item The phrases "SHOULD NOT" or "NOT RECOMMENDED" mean that there might exist valid reasons in particular circumstances when the particular behavior is acceptable or even useful, but the full implications need to be understood and the case carefully weighed before implementing any behavior described with this label.
\item The word "MAY" or the adjective "OPTIONAL" mean that an item is truly optional.
\end{itemize}

\subsection{SBOL Class Names}

The definition of SBOL Visual references several SBOL classes, which are defined as listed here.  For full definitions and explanations, see the SBOL 3.0 data model~\citep{SBOL3_0}.

\begin{itemize}
% Not including: Identified, TopLevel, Sequence

\item \emph{\sbol{Component}}: Describes the structure of designed entities, such as DNA, RNA, and proteins, as well as other entities they interact with, such as small molecules or environmental properties, and the functional relationships and constraints relating these elements.

\item \emph{\sbol{Feature}}:
Represents a specific occurrence or instance of an entity within the design of a \sbol{Component}, such as a promoter in a genetic construct or an enzyme within a synthesis reaction network.

\item \emph{\sbol{SubComponent}}:
A type of \sbol{Feature} that indicates the inclusion of a potentially complicated sub-construct or sub-system within the design of a \sbol{Component}. For example, a \sbol{SubComponent} might represent one gene (with all its attendant complexity) within a multi-gene design or might represent a set of linked reactions within a larger metabolic network.

\item \emph{\sbol{ComponentReference}}:
A type of \sbol{Feature} that is defined by reference to another \sbol{Feature} in a \sbol{SubComponent}. For example, a \sbol{ComponentReference} might be used to indicate a regulation of the promoter in a functional component design included as a \sbol{SubComponent} or a reaction involving the product of a reaction network included as a \sbol{SubComponent}.

% Not including: LocalSubComponent, ExternallyDefined, SequenceFeature

\item \emph{\sbol{Location}}:
Specifies the base coordinates and orientation of a genetic feature on a DNA or RNA molecule or a residue or site on another sequential macromolecule such as a protein.

% Not including: Location sub-classes

\item \emph{\sbol{Constraint}}:
Describes the relative spatial position, sequence orientation, topological relationship, or identity relationship of two \sbol{Feature} objects that are contained within the same \sbol{Component}.

\item \emph{\sbol{Interaction}}:
Describes a functional relationship between \sbol{Feature} objects, such as regulatory activation or repression, or a biological process such as transcription or translation.

\item \emph{\sbol{Participation}}:
Describes the role that a \sbol{Feature} plays in an \sbol{Interaction}.
For example, a transcription factor might participate in an \sbol{Interaction} as a repressor or as an activator.

\item \emph{\sbol{Interface}}:
Describes the intended interface for a \sbol{Component} by designating a set of \sbol{Feature} objects as \sbol{input}, \sbol{output}, or \sbol{nondirectional} elements of the interface.


% Not including: CombinatorialDerivation, Implementation, ExperimentalData, Model, Collection, Attachment
\end{itemize}
